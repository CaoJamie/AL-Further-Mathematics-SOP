\documentclass[]{article}
\usepackage{lmodern}
\usepackage{amssymb,amsmath}
\usepackage{ifxetex,ifluatex}
\usepackage{fixltx2e} % provides \textsubscript
\ifnum 0\ifxetex 1\fi\ifluatex 1\fi=0 % if pdftex
  \usepackage[T1]{fontenc}
  \usepackage[utf8]{inputenc}
\else % if luatex or xelatex
  \ifxetex
    \usepackage{mathspec}
  \else
    \usepackage{fontspec}
  \fi
  \defaultfontfeatures{Ligatures=TeX,Scale=MatchLowercase}
\fi
% use upquote if available, for straight quotes in verbatim environments
\IfFileExists{upquote.sty}{\usepackage{upquote}}{}
% use microtype if available
\IfFileExists{microtype.sty}{%
\usepackage[]{microtype}
\UseMicrotypeSet[protrusion]{basicmath} % disable protrusion for tt fonts
}{}
\PassOptionsToPackage{hyphens}{url} % url is loaded by hyperref
\usepackage[unicode=true]{hyperref}
\hypersetup{
            pdfborder={0 0 0},
            breaklinks=true}
\urlstyle{same}  % don't use monospace font for urls
\IfFileExists{parskip.sty}{%
\usepackage{parskip}
}{% else
\setlength{\parindent}{0pt}
\setlength{\parskip}{6pt plus 2pt minus 1pt}
}
\setlength{\emergencystretch}{3em}  % prevent overfull lines
\providecommand{\tightlist}{%
  \setlength{\itemsep}{0pt}\setlength{\parskip}{0pt}}
\setcounter{secnumdepth}{0}
% Redefines (sub)paragraphs to behave more like sections
\ifx\paragraph\undefined\else
\let\oldparagraph\paragraph
\renewcommand{\paragraph}[1]{\oldparagraph{#1}\mbox{}}
\fi
\ifx\subparagraph\undefined\else
\let\oldsubparagraph\subparagraph
\renewcommand{\subparagraph}[1]{\oldsubparagraph{#1}\mbox{}}
\fi

% set default figure placement to htbp
\makeatletter
\def\fps@figure{htbp}
\makeatother


\date{}

\begin{document}

\section{Further Math Notes}\label{header-n106}

\subsection{Further Pure}\label{header-n108}

\subsubsection{Implicit Differentiation}\label{header-n109}

\begin{enumerate}
\def\labelenumi{\arabic{enumi}.}
\item
  Note, for some implicit differentiation such as \texttt{s03\_01\_07},
  all of the condition in the equation must be consider, such as
  \(x^4 + y^4 = 1\), which is useful to obtain the correct result. Be
  aware of those question
\end{enumerate}

\subsubsection{Series}\label{header-n113}

\begin{enumerate}
\def\labelenumi{\arabic{enumi}.}
\item
  CIE convergence usually related with the general term's convergence:
\end{enumerate}

For a series \(u_1 + u_2 + \dots u_n\), if the general term for
\(\sum_{n=1}^{\infty}{u_n}\) converge then the single term \(u_n\)
converge on a specific limit.

\textbf{Note}, this is not true vice versa, e.g. \(u_n = \frac{1}{n}\)
converge but \(\sum^\infty_{n=1}{u_n}\) doesn't converge

So for question such as \texttt{s03\_01\_3} we can directly obtain the
general term for \(\sum^\infty_{n=1}{u_n}\) and deduce whether the
series converge from it directly

\subsubsection{Differentiation Equation}\label{header-n120}

\begin{enumerate}
\def\labelenumi{\arabic{enumi}.}
\item
  For the complementary function for characteristics having imaginary
  root, \(y_{c.f.} = e^{-\alpha x} (A\cos\beta x + B\sin \beta x)\),
  this formula should be remembered precisely
\end{enumerate}

\subsubsection{Roots of Polynomial}\label{header-n124}

\begin{enumerate}
\def\labelenumi{\arabic{enumi}.}
\item
  For questions such as \texttt{w08\_01\_12} , the last bit usually
  involve some manipulation of the series, which could be done by doing
  algebraic manipulation and note the symmetry property of the
  summation. Usually the objective is to approach some existing
  conclusion for example \(\sum \alpha\) or \(S_n\), therefore we can
  conduct observation on this one:

  \( \alpha^2 (\beta^4 + \gamma^4 + \delta^4) + \beta^2 (\alpha^4 + \gamma^4 + \delta^4) + \gamma^2 (\alpha^4 + \beta^4 +  \delta^4) + \delta^2(\alpha^4 + \beta^4 + \gamma^4)\)
  , it is easy to realize that it is basically similar to \(S_2 S_4\),
  but each \(S_4\) have one term \(\alpha^4\) lost, so to bring it back
  to the polynomial, combinedly it is to obtain that the polynomial
  equals to \(S_2S_4 - S_6\)
\end{enumerate}

\subsubsection{Integration}\label{header-n157}

\begin{enumerate}
\def\labelenumi{\arabic{enumi}.}
\item
  Note that for CIE both 2D and 3D centroid still exists on syllabus so
  it is better to remember both equation, to deal with problem such as
  \texttt{s08\_01\_1}
\end{enumerate}

For 2D centroid:

\[\bar{x} = \cfrac{\int{xy}dx}{\int y dx}, \bar{y} = \cfrac{\frac{1}{2}\int y^2 dx}{\int y dx}\]

For 3D centroid:

\[\bar{x} = \cfrac{\int xy^2 dx}{\int y^2 dx}, \bar{y} = 0\]

the derivation can be dealt with by using the \textbf{finite element
method}

\begin{enumerate}
\def\labelenumi{\arabic{enumi}.}
\item
  when the question ask to obtain a recurrence formula by considering
  some expression such as \(\frac{d}{dx} Q(x)\) the algebraic result
  usually contains two part due to chain rule for differentiation. The
  recurrence formula usually comes from reintegrate the expression
  obtained, but sometimes like \texttt{w08\_01\_07} the answer is not
  immediately obvious. 
\end{enumerate}

\(I_n = \int^1_0 \cfrac{1}{(1+x^4)^n} dx\) , obtain
\(4nI_{n+1} = \frac{1}{2^n} +(4n-1)I_n\) by considering
\(\frac{d}{dx} (\cfrac{x}{(1+x^4)^n})\)

Firstly taking the differentiation gives
\(\cfrac{1}{(1+x^4)^n} - \cfrac{4nx^4}{(1+x^4)^{n+1}}\), but the
expression only match for one term. However, since the denominator is
\((1+x^4)\), we can add and substract 1 from the right hand side part,
therefore resulting
\(\cfrac{1}{(1+x^4)^n} - \cfrac{4n(1+x^4-1)}{(1+x^4)^{n+1}}\) and hence
obtain \(4nI_{n+1} = \frac{1}{2^n} +(4n-1)I_n\)

\subsubsection{Complex Number}\label{header-n133}

\begin{enumerate}
\def\labelenumi{\arabic{enumi}.}
\item
  For questions like \texttt{s03\_01\_06} asking to deduce the roots of
  a given polynomial from the trigonometry expression:
\end{enumerate}

\(\cos(6\theta) = 32\cos^6\theta -48\cos^4\theta + 18\cos^2\theta-1 \Rightarrow 64x^6 - 96x^4 + 36x^2 -1 = 0\)

By F.T.A. we now that the polynomial will have 6 solution, but this is
not the case for the trigonometry identity which works for any
\(\theta\), hence we need to limit some of the expression, by going back
from polynomial we would have:

\(-\frac{1}{2} = 32x^6 - 48x^4 + 18x^2 -1\), if \(x  = \cos\theta\) we
would have \(-\frac{1}{2} = \cos 6\theta = \cos\frac{2}{3}\pi\)

we would easily obtain that

\(6\theta = \pm \frac{2}{3} \pi + 2\pi k, k = 0, \pm1, \pm 2,\dots\)

\(\theta = \pm \cfrac{\pi}{9} + \cfrac{\pi}{3}k\)

\(\theta = \frac{\pi}{9},\frac{2\pi}{9},\frac{4\pi}{9},\frac{5\pi}{9},\frac{7\pi}{9},\frac{8\pi}{9}\)

\textbf{Note} we would want all the roots having different result so be
sure to eliminate the equivalent solution

\begin{enumerate}
\def\labelenumi{\arabic{enumi}.}
\item
\end{enumerate}

\subsubsection{Vector Space}\label{header-n145}

\subsubsection{Vector Geometry}\label{header-n186}

\subsection{Further Mechanics}\label{header-n147}

\subsection{Further Statistics}\label{header-n149}

\subsubsection{P.D.F. and C.D.F.}\label{header-n150}

\begin{enumerate}
\def\labelenumi{\arabic{enumi}.}
\item
  For the substitution of a different variable such as
  \texttt{w02\_02\_10} , i.e. \(Y = X^2\) and obtain the p.d.f. the
  verification for the correctness of the process could be done via
  directly substituting
\end{enumerate}

Given that \(f(x) = \int^\infty_0 p(x) dx\), we would be able to obtain
the p.d.f. for
\(g(y) = \int_0^\infty p(\sqrt(y)) \cfrac{1}{2\sqrt{y}} dy\) by the
means of substitution \(y  = \sqrt{x}\) (Note that the upper limit and
lower limit should also been substituted) so that the result could be
compared with the result from going through the definition of \(G(y)\) ,
but this is only a method for verification

Also, when asking for definition of \(E(x)\) and \(Var(x)\), it is
probably the best by writing the original formula out and then

\end{document}
