\documentclass[]{article}
\usepackage{lmodern}
\usepackage{amssymb,amsmath}
\usepackage{ifxetex,ifluatex}
\usepackage{fixltx2e} % provides \textsubscript
\ifnum 0\ifxetex 1\fi\ifluatex 1\fi=0 % if pdftex
  \usepackage[T1]{fontenc}
  \usepackage[utf8]{inputenc}
\else % if luatex or xelatex
  \ifxetex
    \usepackage{mathspec}
  \else
    \usepackage{fontspec}
  \fi
  \defaultfontfeatures{Ligatures=TeX,Scale=MatchLowercase}
\fi
% use upquote if available, for straight quotes in verbatim environments
\IfFileExists{upquote.sty}{\usepackage{upquote}}{}
% use microtype if available
\IfFileExists{microtype.sty}{%
\usepackage[]{microtype}
\UseMicrotypeSet[protrusion]{basicmath} % disable protrusion for tt fonts
}{}
\PassOptionsToPackage{hyphens}{url} % url is loaded by hyperref
\usepackage[unicode=true]{hyperref}
\hypersetup{
            pdfborder={0 0 0},
            breaklinks=true}
\urlstyle{same}  % don't use monospace font for urls
\usepackage{longtable,booktabs}
% Fix footnotes in tables (requires footnote package)
\IfFileExists{footnote.sty}{\usepackage{footnote}\makesavenoteenv{long table}}{}
\IfFileExists{parskip.sty}{%
\usepackage{parskip}
}{% else
\setlength{\parindent}{0pt}
\setlength{\parskip}{6pt plus 2pt minus 1pt}
}
\setlength{\emergencystretch}{3em}  % prevent overfull lines
\providecommand{\tightlist}{%
  \setlength{\itemsep}{0pt}\setlength{\parskip}{0pt}}
\setcounter{secnumdepth}{0}
% Redefines (sub)paragraphs to behave more like sections
\ifx\paragraph\undefined\else
\let\oldparagraph\paragraph
\renewcommand{\paragraph}[1]{\oldparagraph{#1}\mbox{}}
\fi
\ifx\subparagraph\undefined\else
\let\oldsubparagraph\subparagraph
\renewcommand{\subparagraph}[1]{\oldsubparagraph{#1}\mbox{}}
\fi

% set default figure placement to htbp
\makeatletter
\def\fps@figure{htbp}
\makeatother


\date{}

\begin{document}

\section{Further Mathematics Pure SOP}\label{header-n0}

IMPORTANT NOTES** Although this SOP tried to eliminate all difficulties
to do further math pure question, there comes the necessaries which you
should personally try at least 5-6 set of different past paper to get
familiar with all the question type. The either or question is a
particular hard one and sometimes you may find that they require more
insight than practice. This SOP doesn't responsible for any kind of
mistakes or omission during the production of the notes and suggest the
candidate to do as much as past papers as possible to ensure they truly
understand the approach.

\subsection{Complex Numbers}\label{header-n3}

a complex number could be expressed by
\(a +bi = r(\cos\theta + i\sin\theta)\) for \(r \geq 0\) , which is been
taught as modular argument form in P3( Do notice that familiarity is
required for the complex number part of P3)

By using Euler formula we can express
\(z = r(\cos\theta + i\sin \theta) = re^{i\theta}\)

\subsubsection{Root of Unity}\label{header-n6}

the root of unity can be expressed as:

\(z^n =e^{i\theta*n} =(\cos\theta + i\sin\theta)^n = (\cos n\theta + i\sin n\theta)= 1\)

\(\theta = \frac{2k\pi}{n}\) for \( k=0,1,2,...(n-1)\)

solving other roots would require the additional r in the expression but
the idea is similar

\subsubsection{De Moivre's Theorem}\label{header-n11}

De Moivre's theorem(which you can prove by induction) state that
\((\cos\theta + i\sin\theta)^n = \cos(n\theta) + isin(n\theta)\)

this could be very useful because we can now evaluate:

\(\Re {(\cos\theta+ i\sin\theta)^n} = \cos(n\theta)\) and
\(\Im{(\cos\theta + i\sin\theta)^n} = sin(n\theta)\), which is useful to
simplify the high degree trigonometry expression in integration

Notice that the expression could also have binomial expansion, so some
question would ask you: hence find the relationship for
\(\cos^n(\theta)\) and \(\cos(n\theta)\)

Another frequent testing point is the ability to convert the expression
\((z^n \pm z^{-n})\)

\(z^n + z^{-n} = 2\cos(n\theta)\) and specifically
\(z^1 + z^{-1} = 2\cos(\theta)\)

\(z^n - z^{-n} = 2\sin(n\theta)\) and specifically
\(z^1 - z^{-1} = 2\sin(\theta)\)

Notice you will also need to be able to binomial expand this expression
in order to complete the question

Since the De Moivre's theorem require fluency in trigonometry, so here I
list some might be useful notes:

\begin{itemize}
\item
  \(\sin^2x + \cos^x = 1\)
\item
  \(1 + \tan^2x = \sec^2x\)
\item
  \(1+\cot^2x = \csc^2x\)
\item
  \(\sin(-x) = -\sin(x)\)
\item
  \(\cos(-x) = \cos(x)\)
\item
  \(\tan(-x) = -\tan(x)\)
\item
  \(sin(\frac{\pi}{2} - x) = \cos(x)\) 
\end{itemize}

\subsection{Implicit Differentiation}\label{header-n36}

This is usually the case where it combines with parametric equation, so
take care of doing those questions.

consider an equation:

\(P(y) + P(xy) + P(x) = 0\) where \(P(n)\) represents the polynomial
related to n

by taking \(\frac{d}{dx}[P(y) + P(xy) + P(x) ] = \frac{d}{dx}0\) we
could obtain the result by partial fraction:

\begin{itemize}
\item
  the primary principal is to take y as the variable of x and by chain
  rule multiplying \(\frac{dy}{dx}\) on it
\item
  Notice for the product rule the same principle applies, so be careful
  to deal with the term such as \(P(xy)\)
\end{itemize}

if you want to obtain the second derivatives, the technique is to apply
\(\frac{d}{dx}\) towards the expression you obtain in the first step
again, and plug in the value you obtain for \(\frac{dy}{dx}\) with it.

For parametric equation or the formula that require you to do
substitution, the process could be a little bit more difficult since it
require you to conduct simplification, but remember the core principal
demonstrated before, especially the application of chain rule
\(\frac{dy}{dx} = \frac{dy}{dt} * \frac{dt}{dx}\)

\subsection{Differential Equation }\label{header-n48}

The general solution \(y\) for an differential equation
\(a\ddot{y} + b\dot{y} + cy = f(x)\) can be obtained by finding
\(y = y_{C.F.} + y_{P.I.}\)

\subsubsection{Complimentary Function}\label{header-n50}

For the differential equation \(a\ddot{y} + b\dot{y} + cy = f(x)\) , we
can give the form of C.F. by examine roots of auxiliary equation.

auxiliary equation for the differential equation is
\(ak^2 + bk + c = 0\)

\begin{itemize}
\item
  2 unique real root \(k_1,k_2\), \(y_{C.F.} = A\exp{k_1x}+B\exp{k_2x}\)
\item
  1 unique real root \(k\), \(y_{C.F.} = (A+Bx)\exp{kx}\)
\item
  2 imaginary root \(z = \alpha \pm i\beta \),
  \(y_{C.F.} = \exp{\alpha x}(Asin\beta x + B\cos\beta x)\)
\end{itemize}

\subsubsection{Particular Integral}\label{header-n60}

For the differential equation \(a\ddot{y} + b\dot{y} + cy = f(x)\) ,
find the form of P.I. by giving identical expression of particular
integral \(g(x)\) as \(f(x)\)

\begin{longtable}[]{@{}ll@{}}
\toprule
\(f(x)\) & \(g(x)\)\tabularnewline
\midrule
\endhead
\(A\cos(kx) + B\sin(kx)\) & \(C\cos(kx) + D\sin(kx)\)\tabularnewline
\(A\exp{kx}\) & \(B\exp{kx}\)\tabularnewline
\(Ax^r + Bx^{r-1} + \dots + n\) &
\(Cx^r + Dx^{r-1} + \dots + m\)\tabularnewline
\bottomrule
\end{longtable}

Substituting \(y_{P.I.} = g(x),y'_{P.I.} = g'(x),y''_{P.I.} = g''(x)\)
into the equation
\(a\ddot{y_{P.I.}} + b\dot{y_{P.I.}} + cy_{P.I.} = f(x)\) , solve the
equation by equating the coefficient.

\subsubsection{General Solution and Particular
Solution}\label{header-n76}

Sum the particular integral and complementary function would lead to
general solution y,

Substituting specific starting condition of x,y allow you to give
particular solution

\subsection{Polar Coordinate}\label{header-n79}

Under the SOP, the notation will be denoted with
\((x,y)_{c} \longmapsto(r,\theta)_{p}\) as c stands for Cartesian and p
stands for polar

\subsubsection{Transformation}\label{header-n81}

Transform the coordinate from Cartesian to Polar:

\((x,y)_{c} \equiv (\sqrt{x^2 + y^2},\tan^{-1} \cfrac{y}{x})_{p}\)

Transform the coordinate from Polar to Cartesian:

\((r,\theta)_{p} \equiv (r\cos \theta, r\sin \theta)_{c}\)

\subsubsection{Sketch the graph}\label{header-n86}

In the polar coordinate, the positive x axis is called initial line, and
\(\theta\) is the angle between initial line and the positive axis
anticlockwise

for expression \(r = f(\theta)\) take the table of the following value,
plug in and calculate the result for r:

\(\theta :=  0,\frac{\pi}{6},\frac{\pi}{4},\frac{\pi}{3},\frac{\pi}{2},\frac{2\pi}{3},\frac{3\pi}{4},\frac{5\pi}{6},\pi,-\pi,-\frac{5\pi}{6},-\frac{3\pi}{4},-\frac{2\pi}{3},-\frac{\pi}{2},-\frac{\pi}{3},-\frac{\pi}{4},-\frac{\pi}{6}\)

Notice this should give a symmetrical \(\theta\) samples around the
axis, each quadrant includes 5 samples point(including points on axis),
and commonly this should allow you to take the graph accurately enough

If taking two graph, calculate the potential position for which the
graph are contact by equating \(C_{1} = C_{2}\)

If asked, takes the calculation of any stationary point by taking
\(\cfrac{dr}{d\theta}\)

After getting results, neglect any negative r as definition for CIE
don't require to sketch the r graph, and the domain for
\(\theta \in (-\pi,\pi]\), but the question may asked for different
value

\subsubsection{Noticeable Graphs}\label{header-n94}

\begin{itemize}
\item
  Circle: \(r = a, r = a\sin\theta, r = a\cos\theta\)
\item
  Straight Line: \(r = a\sec\theta, r = a\csc\theta\)
\item
  Spiral:\(r = a\theta\)
\end{itemize}

\subsubsection{Common Symmetry}\label{header-n102}

\begin{itemize}
\item
  \(sin\theta\) is symmetric about \(\theta = \frac{\pi}{2}\)
\item
  \(\cos\theta\) is symmetric about \(\theta = 0\)
\end{itemize}

Notice \(\theta\) can be extend to any expression within those
trigonometry identities

\subsubsection{Arc Length}\label{header-n109}

The arc Length of the curve in polar coordinate on \(r = f(\theta)\)
between \((r_1,\theta_1)\) and \((r_2,\theta_2)\) is given by the
formula:

\(l = \int^{\theta_1}_{\theta_2} \sqrt{r^2 + (\cfrac{dr}{d\theta})^2}d\theta\)

Notice the curve must be continuous between the upper and lower bound
for the integral

\subsubsection{Area}\label{header-n113}

The area enclosed by the curve \(r = f(\theta)\) between
\((r_1,\theta_1)\) and \((r_2,\theta_2)\) is

\(A= \cfrac{1}{2}\int^{\theta_1}_{\theta_2} r^2d\theta\)

Notice the curve must be continuous between the upper and lower bound
for the integral

\subsection{Further Integration}\label{header-n117}

\subsubsection{Integration Skills}\label{header-n118}

Apart from integration by part, substitution(happily most of the
substitution in further math would come with the question so at least
you would know your starting point) and partial fraction(which you
should be familiar already in P3), some of the additional skills in
further math is required:

sometimes you would notice there exist fraction that looks like these:

\(\int \cfrac{1}{\sqrt{a^2 - x^2}}dx = arcsin(\cfrac{x}{a})\) which is
clearly demonstrated to solve by having substitution of \(x = a sin(x)\)

and then solve with trigonometry identity \(sin^2{x} + cos^2{x}=1\)

\(\int{\cfrac{1}{a^2 + x^2}}dx = \frac{1}{a}arctan(\frac{x}{a})\) which
is similarly demonstrated to solve by having subsitution of
\(x = a tan(x)\) and then solve with trigonometry identity
\(1+tan^2{x} = sec^2{x}\)

if you find you are unfamiliar with this part of the context, reference
to MF10 formula sheet for help

and some small reminder from P3(you should recall using partial fraction
to solve this question):

\(\int{\cfrac{1}{a^2 - x^2}}dx = \frac{1}{2a} ln \left|{\cfrac{a+x}{a-x}}\right|\)

\(\int{\cfrac{1}{x^2-a^2}}dx = \frac{1}{2a} ln \left|\cfrac{x-a}{x+a}\right|\)

\subsubsection{Reduction Formula}\label{header-n128}

This question is very characteristic in the further math test because
usually the question will contains the expression such as
\(I_n= \int f(x) * g(x)'dx\)

In this case the expression is usually solve by using integration by
parts which will then give the formula in

so what we need to do is to takes integration by parts and obtain:

\(I_n = [f(x)g(x)] - \int{f(x)'g(x)}dx\), where the expression usually
evaluate \(\int f(x)'g(x) dx\) to another \(I_k\) related to \(I_n\),
which you could then related the reduction formula
to\(I_n = f(x) + P(x)I_k\), or something similar to this.

The question usually goes without a hence, but sometimes you would be
ask to derive some expression and then start working from there. For the
expression that contains derivatives it is not a bad idea to try at
least integrate the both side.

Usually the question would ask you to demonstrate a certain value for
\(I_n\), such as \(I_5\), in which you should demonstrate all the
process for calculating intermediate \(I_k\) to prevent mark loss

\subsubsection{Mean Value}\label{header-n135}

This is required by the syllabus(although not tested as often as you
might expected). The actual technique usually only worth a small portion
of your mark, but the formula should be remember by heart:

\(\bar{f} = \cfrac{1}{b-a} \int_a^b f(x) dx\)

\subsubsection{Arc Length}\label{header-n138}

The arclength for a equation \(y = f(x)\) could be expressed by:

\(l = \int{\sqrt{1 + (\frac{dy}{dx})^2} dx}\)

when encountering parametric equation, you could change the equation
into:

\(l = \int{\sqrt{{\frac{dy}{dt}}^2 + {\frac{dx}{dt}}^2}}dt\)

notice in the case of parametric equation you need to check the upper
and lower bond to ensure that the limit had been changed from x to t.

Plug in the number and you would obtain the answer. Notice due to the
appearance of square root, it is usual that you can simplify this using
the skills of polynomial or by using the trigonometry identities(which
is more often to appear compare with the previous one)

\subsubsection{Surface Area}\label{header-n145}

The surface area of a curve enclosed by the axis could be expressed as
this:

Enclosed with x axis: \(A = \int 2\pi y ds\), where
\(ds = \sqrt{1+{\frac{dy}{dx}}^2}dx\)

Enclosed with y axis: \(A = \int 2\pi x ds\), where
\(ds = \sqrt{1+{\frac{dx}{dy}}^2}dy\)

notice for the rotation around the y axis would require you to take the
inverse function so that \(x = h(y)\)

\subsubsection{Centroid}\label{header-n150}

These questions are rarely asked, but when they appeared it is usually
in one of those either or question.

It's still on the syllabus and there were records where this knowledge
was required to apply into the question 1-10. I suggest taking some time
to remember the formula in case you need it.

For 2D region:

\(\bar{x} = \cfrac{\int{xy}dx}{\int{y}dx}\) and
\(\bar{y} = \cfrac{\frac{1}{2}\int{y^2}dx}{\int{y}dx}\)

Notice in the case of a parametric equation you may need to substitute
\(dx\) with expression of \(dt\) by using \(dx = \frac{dx}{dt} * dt\)
and \textbf{changing} the upper and lower bond with the appropriate t
value

For 3D region (which is even less likely to be tested):

\(\bar{x} = \cfrac{\int_v{x}dv}{\int_v dv}\) and
\(\bar{y} = \cfrac{\int_v{y}dv}{\int_v dv}\)

\subsection{Curve Sketching}\label{header-n158}

Here are the standard procedure to sketch the curve (which you should
follow at most of the time)

\subsubsection{Horizontal and Vertical Asymptote}\label{header-n160}

Notice that usually the first few mark is awarded towards determine the
asymptote.

In this case you should easily obtain the horizontal asymptote for
\(y = f(x)\) by calculating the following limit:

\(\lim_{x->\infty} y \) and \(\lim_{x-> - \infty} y \)

simpliy divide the highest degree from the above part of the fraction
and obtain the answer. Notice if the upper part is higher than it tends
to \(\infty\) and if it is lower than it tends to 0. If the degree is
the same then it tends to the coefficient of the highest degree

notice in most of the time the result would provide the same, but you do
need to double check it to make sure it is actually true

For the vertical asymptote, I suggest you only search them when the
formula for \(y = f(x)\) could be simplify into
\(f(x) = \cfrac{h(x)}{g(x)}\), which in this case you should obtain the
result by calculating for \(g(x) = 0\), i.e. the lower part of the
fraction equals 0

\subsubsection{Oblique Asymptote}\label{header-n167}

Notice that in some of the exam you would found that the
expression\(y = f(x)\) could be simplify to the following result
\(f(x) = \cfrac{x^n+1 + x^n + \dots + c}{x^n + \dots + 1}\), which means
that the higher part of the fraction is a polymer of x that is one
degree higher than the lower part, you should not seek the horizontal
asymptote but instead search for the oblique asymptote in the form of
\(y = ax+b\)

so consider the expression

\(y = \cfrac{ax^2 + bx + c}{ex+f}\)

we want to evaluate its into the form \(y = mx + k + O(1/x)  +\dots\)

hence we would use the long division in this case and only take the
terms for \(mx + k\) since the rest is 0 under the limit as
\(x \rightarrow \infty\)

\subsubsection{Stationary Point}\label{header-n173}

The next important set of point to consider is the stationary point.
This should be simple to obtain because you would only need to find the
point for \(\frac{dy}{dx} = 0\)

Remember to indicate those point on the sketch and state that they are
the stationary point

In some cases you may need to find the nature of the stationary point by
evaluating \(\frac{d^2y}{dx^2}\)

recall that:

\(\frac{d^2y}{dx^2} > 0 \implies minimum\)

\(\frac{d^2y}{dx^2} = 0 \implies point \space inflextion\)

\(\frac{d^2y}{dx^2} < 0 \implies maximum\)

\subsubsection{Intersection with Axes}\label{header-n181}

The x intersection and y intersection is always required for sketching,
by simply plug in \(y=f(x)\) with:

\(f(0) \rightarrow (0,f(0))\) and \(f(k) = 0 \rightarrow (k,0)\)

\subsubsection{Sketching Techniques}\label{header-n184}

after drawing the dash line for the asymptote and label stationary
point, here is a useful trick for quick sketching:

starting from\(-\infty\) , consider the sign of the expression
\(f(x) = \frac{h(x)}{g(x)}\) , by using the point
\(k+\epsilon, k-\epsilon\) around the point \(x = k\) so that you would
have a rough understanding of the trend of the function

Combine with the previous result you could sketch the curve easily

\subsection{Series}\label{header-n188}

\subsubsection{Standard Result}\label{header-n189}

\(\sum{r} =\frac{1}{2}n(n+1)\)

\(\sum r^2 = \frac{1}{6}n(n+1)(n+2)\)

\(\sum r^3 = \frac{1}{4}n^2(n+1)^2\)

by separating the summation into the linear combination of the standard
result you would be able to easily obtain a result

\subsubsection{Method of Difference}\label{header-n194}

if you see some expression such as \(\sum f(p) - f(q)\) which in
particular include form such as partial fraction, or other similar
patterns

you can expand it to something similar to this(the precise number is
based on \textbf{question}):

\(= f(1) - f(2) - f(2) + f(3) -\dots + f(n-1) - f(n)\)

and hence you would get by elimination

\( = f(1) - f(n)\), which gives the simplify solution to the question

the pattern is that there is consecutive \(f(n+k) - f(n-k)\) appears
into the extended form, and usually the question would give you a hint
if the method is relatively difficult to found

\subsubsection{Convergence}\label{header-n201}

the degree of convergence which CIE requires is by analyzing the
analytic solution to the sums as \(n \rightarrow \infty\) and see if
this converge to a certain number.

Specifically, to consider the convergence of a series, you can do this:

\(\lim_{n\rightarrow \infty} \sum^n_{r=1} a^r\)

Check if that analytic formula tends to a specific limit.(a small trick
would be to at least check \(lim_{n\rightarrow \infty} a^n\) has a limit
otherwise the series diverge)

\subsection{Polynomials}\label{header-n206}

\subsubsection{Relations Between Roots and
Coefficient}\label{header-n207}

For a general polynomial
\(P(x) = ax^n + bx^{n-1} + cx^{n-2} + \dots + d\) , it will have root
\(\alpha, \beta, \gamma \dots\), then we would be able to obtain the
following result easily:

\begin{itemize}
\item
  \(\sum \alpha = -\cfrac{b}{a}\)
\item
  \(\sum \alpha\beta = \cfrac{c}{a}\)
\item
  \(\sum \alpha \beta\gamma = -\cfrac{d}{a}\)
\item
  \(\dots\) Notice the pattern can be extended to even higher degree by
  repeating this pattern for a specific equation
\end{itemize}

These results should be carefully remembered. Notice that the sum
notation actually means symmetrical sums, which means all of the
possible permutation of the variable should be sum. For example,
\(\sum \alpha \beta \) means actually to have
\(\alpha \beta + \alpha \gamma + \beta \gamma\) So make sure you
calculate the sum correctly. By summing these value and getting the
equation together, we can find the value of coefficient.

Other than that there are other formula (Newton's Sum) that would be
useful in the context of questions, which you should remember
clearly(Notice MF10 doesn't contain those formula so you should be
caution to remember them clearly)

\begin{itemize}
\item
  \(S_1 = \sum \alpha = -\cfrac{b}{a}\)
\item
  \(S_2 = \sum{\alpha}^2 = (\sum \alpha)^2 - 2\sum \alpha \beta\)
\item
  \(S_3=\sum \alpha^3 = (\sum \alpha)^3 - 3\sum \alpha \beta\sum\alpha+3\sum \alpha\beta\gamma\)
\item
  \(S_4=\sum \alpha^4  = (\sum \alpha)^4 - 4\sum \alpha\beta(\sum \alpha)^2 +4\sum\alpha\beta\gamma\sum\alpha+2(\sum\alpha\beta)^2 - 4\sum\alpha\beta\gamma\delta\)
  {[}usually we would recommend that you obtain it via substitution or
  recurrence instead due to the difficulties for remembering it{]}
\item
  \( S_{-1}=\sum \cfrac{1}{\alpha} = \cfrac{\sum \alpha \beta}{\sum \alpha \beta\gamma}\)
\end{itemize}

\subsubsection{Method of Substitution}\label{header-n231}

During the exam, there exists some question which ask you to find for
example, equation suitable for
\(\cfrac{1}{\alpha},\cfrac{1}{\beta},\cfrac{1}{\gamma}\) or similar
component form from the original roots of the equation. Otherwise, the
substitution would be directly given and you will be required to carry
out algebraic manipulation to form a new equation, by for example,
substituting for instance, \(y = \cfrac{1}{x}, y = x^2,  y = x^4\) into
the original polynomial of \(P(x)\) and obtain a new equation from it.

Notice sometimes you would have equation containing \(y^\frac{1}{2}\),
in this way you should move all the integer power to one side and
fractional power to other side, having suitable power(for example power
of 2 in this case) for both side and hence get the new equation.

\subsubsection{Method of Recurrence}\label{header-n234}

given a polynomial \(P(x) = ax^n + bx^{n-1} + cx^{n-2} + \dots + d\), we
know by \textbf{factor theorem} that if
\(x = \alpha,\beta,\gamma,\dots\) is the root of polynomial, then:

\begin{itemize}
\item
  \(a\alpha^n + b\alpha^{n-1} + c\alpha^{n-2} + \dots + d = 0\)
\item
  \(a\beta^n + b\beta^{n-1} + c\beta^{n-2} + \dots + d = 0\)
\item
  \(a\gamma^n + b\gamma^{n-1} + c\gamma^{n-2} + \dots + d = 0\)
\item
  \(\dots\)
\end{itemize}

summing all of them, we would easily obtain the relationship:
\(aS_n + bS_{n-1} + cS_{n-2} + \dots + n*d = 0\)

this allow us to have another approach to obtain \(S_n\)

Notice, this relationship
\(aS_n + bS_{n-1} + cS_{n-2} + \dots + n*d = 0\) can be extend to any
power of S, for example the same equation implies also
\(aS_{n-1} + bS_{n-2} + cS_{n-3} + \dots + n*dS_{-1} = 0\) since you
would find that dividing a x from the original equation means the
formula will still hold.

\subsection{Proof by Induction}\label{header-n248}

Proof by induction follow a standard procedure which you should clearly
demonstrate your understanding of this format onto the paper:

\begin{enumerate}
\def\labelenumi{\arabic{enumi}.}
\item
  State the statement: \(H_k : P(k)\)
\item
  show that \(H_1\) is true
\item
  assume \(H_k\) is true for some number \(n\in \N\) or \(n \in \Z^+\)
\item
  show that \(P(n+1)\) is true by the assumption that \(H_n\) is ture
\item
  state that \(H_n \implies H_{n+1}\)
\item
  state that by the principal of mathematical induction(PMI), the
  statement is true for all \(n\in \N\) or \(n \in \Z^+\)
\end{enumerate}

Notice most of the mathematical induction question would be related to
number theory or calculus expression, which you should be able to solve
with some algebra manipulation(such as taking \(n = a*k\) for those
divisible question), but for some of those series question, you may need
to think it carefully or seek help directly from the Maclaurin's
expansion in MF10

\subsection{Vector Geometry}\label{header-n264}

\subsubsection{Vector Product}\label{header-n265}

\(|a\times b| = |a||b|\hat{n}\sin\theta \)

\(a \times b = \begin{vmatrix} i&&j&&k\\a_1&&a_2&&a_3\\b_1&&b_2&&b_3\end{vmatrix}\)

notice that the usual way of having cross product from the vector
derives from this notation, and it is recommended that candidate use
this expression instead for easier memorization

hence, the area of the triangle form from these two vectors is
\(\cfrac{1}{2} |a\times b|\)

\subsubsection{Vector Equation of Line}\label{header-n270}

Vector equation of line can be expressed as
\(\underline{r} = \underline{a}+\lambda \underline{b}\), representing a
line pointing from a given point \(\underline{a}\) to all direction on
\(\underline{b}\)

\paragraph{Finding shortest line from a point to a
line}\label{header-n272}

The shortest distance is always perpendicular towards the line from the
point, hence for a point P to a line r, we would derive it from
\(PQ\bullet \underline{r} = 0\), where Q is the point on the line
forming from the line expression

\paragraph{Finding shortest distance from two skew
lines}\label{header-n274}

The shortest distance between
\(\underline{r_1} = \underline{a_1}+\lambda_1 \underline{b_1}\) and
\(\underline{r_2} = \underline{a_2}+\lambda_2 \underline{b_2}\) is given
by the formula
\(|\cfrac{(a_2-a_1)\bullet(b_1\times b_2)}{|b_1 \times b_2|}|\)

\subsubsection{Planes}\label{header-n276}

three forms of planes equations:

\begin{itemize}
\item
  the scalar product equation \(\underline{r}\bullet\underline{n} = p\) 
\item
  the cartesian equation \(ax + by+ cz = d\)
\item
  the vector equation
  \(\underline{r} = \underline{a} + s\underline{b} + t\underline{c}\)
\end{itemize}

\paragraph{Finding the angle between two plane}\label{header-n285}

for \(\underline{r_1}\bullet\underline{n_1} = d_1\) and
\(\underline{r_2}\bullet\underline{n_2} = d_2\), the angle between the
two plane can be simply given as
\(\cos{\alpha} = \cfrac{\underline{n_1}\bullet\underline{n_2}}{|\underline{n_1}||\underline{n_2}|}\)

notice there exists two angle: \(\alpha\) and \(180^{\circ} - \alpha\),
you should read through the question carefully and decide whether this
is acute angle or not.(NB usually it will be acute angle)

\paragraph{Finding the line of intersection between two
plane}\label{header-n288}

for \(\underline{r_1}\bullet\underline{n_1} = d_1\) and
\(\underline{r_2}\bullet\underline{n_2} = d_2\), the line of
intersection is obtained by first find \(\underline{n_c}\) and obtain a
point of the new line by having two cartesian equation
\(ax_1 + by_1+ cz_1 = d_1\) and \(ax_2 + by_2+ cz_2 = d_2\) and setting
one variable as free variable to solve the set of equation, for example
\(x = 1\)

if the equation can't be solved, change a free variable

\paragraph{Angle between a line and a plane}\label{header-n291}

for a line \(\underline{r} = \underline{a} + \lambda\underline{b}\) and
a plane \(\underline{r}\bullet\underline{n} = p\) the angle between the
line and the plane can be calculated via:

\(\cos{\alpha} = \cfrac{n\bullet b}{|n||b|}\) and then obtain the angle
via \(90^{\circ} - \alpha\) or \(\alpha - 90^{\circ} \) depend on the
size of the angle(because n is perpendicular to the plane)

\paragraph{Shortest distance between a point and a
plane}\label{header-n294}

for a point P and a plane \(\underline{r}\bullet\underline{n} = p\) the
distance should be \(|\cfrac{\overrightarrow{\rm PQ}\bullet n}{|n|}|\),
where Q is any point on the plane(usually obtained through solving the
cartesian equation for the plane by setting two variable to 0 and hence
gives x)

\subsection{Matrix and Determinant}\label{header-n296}

\subsection{Vector Space}\label{header-n298}

Vector space is a relatively abstract topic in the further math so it is
consider difficult first encounter. However, CIE also consider about
this situation so the question usually set is easier to solve compare
with the other exam board.

\subsubsection{Axioms of Vector Space}\label{header-n300}

\subsubsection{linear independent}\label{header-n301}

\subsubsection{subspace}\label{header-n302}

\subsubsection{Spanning a space}\label{header-n303}

\subsubsection{Range space}\label{header-n304}

\(\Epsilon \Delta\Xi\)

\end{document}
